\documentclass[a4paper, 12pt, oneside]{article}
\usepackage[utf8]{inputenc}
\usepackage[russian]{babel}
\usepackage{amsmath}
\usepackage[thinc]{esdiff}
\usepackage{mathtools}
\usepackage[T1]{fontenc}
\usepackage[top=26mm, bottom=20mm]{geometry}

\title{Ковариантные производные тензорного поля, заданного в цилиндрических координатах.}
\author{Очкин Никита Валерьевич ФН11-42Б}

\begin{document}

\maketitle
\subsection*{Условие:}
Задано тензорное поле $\mathrm{T}$($X^i$), где $X^i$ - цилиндрические координаты. Найти:\\
$\mathrm{1)}$ ковариантные, контравариантные компоненты этого поля в базисах $\mathrm{r_i}$, $\mathrm{r^i}$,
где $\mathrm{r_i}$ — ортогональный локальный базис цилиндрической системы координат;\\
$\mathrm{2)}$ ковариантные производные компонент тензорного поля в базисах $\mathrm{r_i}$, $\mathrm{r^i}$\\
\subsection*{Исходные данные:}
Поле задано в виде: $\mathrm{T}$ = $T^ij$$\mathrm{e_i}$$\otimes$$\mathrm{e_j}$;\\

\[
T^{ij}=\begin{pmatrix}
	0 & -(X^2) + (X^1) & 0\\
	0 & 0 & 0\\
	2\cdot (X^3) & 0 & 0
\end{pmatrix}
\]

\subsection*{Решение:}
$\mathrm{r_i}$ - ортогональный локальный базис цилиндрической системы координат:\\
\[
\begin{cases}
  x^1 = X^1 \cdot \cos(X^2)\\
  x^2 = X^1 \cdot \sin(X^2)\\
  x^3 = X^3
\end{cases}
\]\\
$\mathrm{1)}$ Для того, чтобы найти компоненты тензорного поля в базисах $\mathrm{r_i}$, $\mathrm{r^i}$,
найдем сначала метрическую матрицу для цилиндрических координат $X^i$ и локальные векторы базиса.\\
$\mathrm{1.1)}$ Найдем якобиеву матрицу для криволинейных координат $X^i$:\\
\[
Q^i_k = \frac{\partial x^i}{\partial X^k}
\]
\begin{alignat*}{3}
  & Q^1_1 = \frac{\partial x^1}{\partial X^1} = \cos((X^2)) \quad &&Q^1_2 = \frac{\partial x^1}{\partial X^2} = -(X^1)\cdot \sin((X^2)) \quad &&Q^1_3 = \frac{\partial x^1}{\partial X^3} = 0 \\
  & Q^2_1 = \frac{\partial x^2}{\partial X^1} = \sin((X^2)) \quad &&Q^2_2 = \frac{\partial x^2}{\partial X^2} = (X^1)\cdot \cos((X^2)) \quad &&Q^2_3 = \frac{\partial x^2}{\partial X^3} = 0 \\
  & Q^3_1 = \frac{\partial x^3}{\partial X^1} = 0 \quad &&Q^3_2 = \frac{\partial x^3}{\partial X^2} = 0 \quad &&Q^3_3 = \frac{\partial x^3}{\partial X^3} = 1 
\end{alignat*}\\
Тогда якобиева матрица цилиндрической системы координат имеет вид:\\

\[
Q^i_k=\begin{pmatrix}
	\cos((X^2)) & -(X^1)\cdot \sin((X^2)) & 0\\
	\sin((X^2)) & (X^1)\cdot \cos((X^2)) & 0\\
	0 & 0 & 1
\end{pmatrix}
\]\\
$\mathrm{1.2)}$ Найдем локальные векторы базиса для цилиндрических координат $X^i$:\\
\[
\mathrm{r_k} = \frac{\partial x^i}{\partial X^k}\overline{\mathrm{e_k}} = Q^i_k\overline{\mathrm{e_k}}
\]\\
Матрица $Q^i_k$ была найдена на предыдущем шаге.\\
$\mathrm{1.3)}$ Найдем метрическую матрицу для цилиндрических координат $X^i$:\\
\[
\mathrm{g_{{ij}}} = \mathrm{r_i}\cdot\mathrm{r_j} = Q^s_iQ^p_j\delta_{sp}
\]\\
Найдем компоненты метрической матрицы для цилиндрической системы координат:\\
\begin{alignat*}{1}
  & \mathrm{g_{11}} = Q^s_1 \cdot Q^p_1 \cdot \delta_{sp} = 1 \\
  & \mathrm{g_{12}} = Q^s_1 \cdot Q^p_2 \cdot \delta_{sp} = 0 \\
  & \mathrm{g_{13}} = Q^s_1 \cdot Q^p_3 \cdot \delta_{sp} = 0 \\
  & \mathrm{g_{21}} = Q^s_2 \cdot Q^p_1 \cdot \delta_{sp} = 0 \\
  & \mathrm{g_{22}} = Q^s_2 \cdot Q^p_2 \cdot \delta_{sp} = (X^1)^2 \\
  & \mathrm{g_{23}} = Q^s_2 \cdot Q^p_3 \cdot \delta_{sp} = 0 \\
  & \mathrm{g_{31}} = Q^s_3 \cdot Q^p_1 \cdot \delta_{sp} = 0 \\
  & \mathrm{g_{32}} = Q^s_3 \cdot Q^p_2 \cdot \delta_{sp} = 0 \\
  & \mathrm{g_{33}} = Q^s_3 \cdot Q^p_3 \cdot \delta_{sp} = 1 
\end{alignat*}\\
Запишем полученную метрическую матрицу для цилиндрической системы координат:\\
\[
g_{ij}=\begin{pmatrix}
	1 & 0 & 0\\
	0 & (X^1)^2 & 0\\
	0 & 0 & 1
\end{pmatrix}
\]\\
Найдем обратную метрическую матрицу:\\
\[
g^{ij}=\begin{pmatrix}
	1 & 0 & 0\\
	0 & \frac{1}{(X^1)^2} & 0\\
	0 & 0 & 1
\end{pmatrix}
\]\\
$\mathrm{1.4)}$ Найдем векторы взаимного локального базиса для цилиндрических координат $X^i$:\\
\[
\mathrm{r^{i}} = \mathrm{g^{{ij}}}\mathrm{r_j} = \mathrm{g^{{ij}}}Q^m_j\overline{\mathrm{e_m}} = Q^{im}\overline{\mathrm{e_m}}:
\]\\
\begin{alignat*}{1}
  & Q^{11} = \mathrm{g^{1j}}Q^1_j = g^{11} \cdot Q^1_1+ g^{12} \cdot Q^1_2= g^{13} \cdot Q^1_3= \cos((X^2))+ 0+ 0= \cos((X^2)) \\
  & Q^{12} = \mathrm{g^{1j}}Q^2_j = g^{11} \cdot Q^2_1+ g^{12} \cdot Q^2_2= g^{13} \cdot Q^2_3= \sin((X^2))+ 0+ 0= \sin((X^2)) \\
  & Q^{13} = \mathrm{g^{1j}}Q^3_j = g^{11} \cdot Q^3_1+ g^{12} \cdot Q^3_2= g^{13} \cdot Q^3_3= 0+ 0+ 0= 0 \\
  & Q^{21} = \mathrm{g^{2j}}Q^1_j = g^{21} \cdot Q^1_1+ g^{22} \cdot Q^1_2= g^{23} \cdot Q^1_3= 0+ -\sin((X^2))/(X^1)+ 0= -\sin((X^2))/(X^1) \\
  & Q^{22} = \mathrm{g^{2j}}Q^2_j = g^{21} \cdot Q^2_1+ g^{22} \cdot Q^2_2= g^{23} \cdot Q^2_3= 0+ \cos((X^2))/(X^1)+ 0= \cos((X^2))/(X^1) \\
  & Q^{23} = \mathrm{g^{2j}}Q^3_j = g^{21} \cdot Q^3_1+ g^{22} \cdot Q^3_2= g^{23} \cdot Q^3_3= 0+ 0+ 0= 0 \\
  & Q^{31} = \mathrm{g^{3j}}Q^1_j = g^{31} \cdot Q^1_1+ g^{32} \cdot Q^1_2= g^{33} \cdot Q^1_3= 0+ 0+ 0= 0 \\
  & Q^{32} = \mathrm{g^{3j}}Q^2_j = g^{31} \cdot Q^2_1+ g^{32} \cdot Q^2_2= g^{33} \cdot Q^2_3= 0+ 0+ 0= 0 \\
  & Q^{33} = \mathrm{g^{3j}}Q^3_j = g^{31} \cdot Q^3_1+ g^{32} \cdot Q^3_2= g^{33} \cdot Q^3_3= 0+ 0+ 1= 1 
\end{alignat*}\\
Запишем полученную матрицу:\\
\[
Q^{im}=\begin{pmatrix}
	\cos((X^2)) & \sin((X^2)) & 0\\
	-\sin((X^2))/(X^1) & \cos((X^2))/(X^1) & 0\\
	0 & 0 & 1
\end{pmatrix}
\]\\
$\mathrm{1.5)}$Найдем теперь компоненты тензорного поля.\\
По условию\\
\begin{alignat*}{3}
  & T^{11} = 0 \quad &&T^{12} = -(X^2) + (X^1) \quad &&T^{13} = 0 \\
  & T^{21} = 0 \quad &&T^{22} = 0 \quad &&T^{23} = 0 \\
  & T^{31} = 2\cdot (X^3) \quad &&T^{32} = 0 \quad &&T^{33} = 0 
\end{alignat*}\\
Вычислим ковариантные компоненты по формуле:\\
\[
T_{ij} = T^{kl}g_{ki}g_{lj};
\]\\
\begin{alignat*}{1}
  & T_{11} = T^{kl}\mathrm{g_{k1}g_{l1} = }0 \\
  & T_{12} = T^{kl}\mathrm{g_{k1}g_{l2} = }(X^1)^2\cdot (-(X^2) + (X^1)) \\
  & T_{13} = T^{kl}\mathrm{g_{k1}g_{l3} = }0 \\
  & T_{21} = T^{kl}\mathrm{g_{k2}g_{l1} = }0 \\
  & T_{22} = T^{kl}\mathrm{g_{k2}g_{l2} = }0 \\
  & T_{23} = T^{kl}\mathrm{g_{k2}g_{l3} = }0 \\
  & T_{31} = T^{kl}\mathrm{g_{k3}g_{l1} = }2\cdot (X^3) \\
  & T_{32} = T^{kl}\mathrm{g_{k3}g_{l2} = }0 \\
  & T_{33} = T^{kl}\mathrm{g_{k3}g_{l3} = }0 
\end{alignat*}\\
Запишем полученную матрицу:\\
\[
T_{ij}=\begin{pmatrix}
	0 & (X^1)^2\cdot (-(X^2) + (X^1)) & 0\\
	0 & 0 & 0\\
	2\cdot (X^3) & 0 & 0
\end{pmatrix}
\]\\
\[
T^i_j = T^{ik}g_{kj};
\]\\
\begin{alignat*}{1}
  & T^1_1 = T^{13}\mathrm{g_{k1} = }0 \\
  & T^1_2 = T^{13}\mathrm{g_{k2} = }(X^1)^2\cdot (-(X^2) + (X^1)) \\
  & T^1_3 = T^{13}\mathrm{g_{k3} = }0 \\
  & T^2_1 = T^{23}\mathrm{g_{k1} = }0 \\
  & T^2_2 = T^{23}\mathrm{g_{k2} = }0 \\
  & T^2_3 = T^{23}\mathrm{g_{k3} = }0 \\
  & T^3_1 = T^{33}\mathrm{g_{k1} = }2\cdot (X^3) \\
  & T^3_2 = T^{33}\mathrm{g_{k2} = }0 \\
  & T^3_3 = T^{33}\mathrm{g_{k3} = }0 
\end{alignat*}\\
Запишем полученную матрицу:\\
\[
T^i_j=\begin{pmatrix}
	0 & (X^1)^2\cdot (-(X^2) + (X^1)) & 0\\
	0 & 0 & 0\\
	2\cdot (X^3) & 0 & 0
\end{pmatrix}
\]\\
\[
T^j_i = T^{kj}g_{ki};
\]\\
\begin{alignat*}{1}
  & T^1_1 = T^{31}\mathrm{g_{k1} = }0 \\
  & T^2_1 = T^{32}\mathrm{g_{k1} = }-(X^2) + (X^1) \\
  & T^3_1 = T^{33}\mathrm{g_{k1} = }0 \\
  & T^1_2 = T^{31}\mathrm{g_{k2} = }0 \\
  & T^2_2 = T^{32}\mathrm{g_{k2} = }0 \\
  & T^3_2 = T^{33}\mathrm{g_{k2} = }0 \\
  & T^1_3 = T^{31}\mathrm{g_{k3} = }2\cdot (X^3) \\
  & T^2_3 = T^{32}\mathrm{g_{k3} = }0 \\
  & T^3_3 = T^{33}\mathrm{g_{k3} = }0 
\end{alignat*}\\
Запишем полученную матрицу:\\
\[
T^j_i=\begin{pmatrix}
	0 & -(X^2) + (X^1) & 0\\
	0 & 0 & 0\\
	2\cdot (X^3) & 0 & 0
\end{pmatrix}
\]\\
Найдем символы Кристоффеля по формуле:\\
\[
\Gamma^m_{ij} = \frac{1}{2}g^{km}(\frac{\partial g_{kj}}{\partial X^i} + \frac{\partial g_{ik}}{\partial X^j} - \frac{\partial g_{ij}}{\partial X^k}):
\]\\
При m = 1:\\
\begin{alignat*}{1}
  & \Gamma^1_{11} = \frac{1}{2}g^{k1}(\frac{\partial g_{k1}}{\partial X^1} + \frac{\partial g_{1k}}{\partial X^1} - \frac{\partial g_{11}}{\partial X^k}) = 0 \\
  & \Gamma^1_{12} = \frac{1}{2}g^{k1}(\frac{\partial g_{k2}}{\partial X^1} + \frac{\partial g_{1k}}{\partial X^2} - \frac{\partial g_{12}}{\partial X^k}) = 0 \\
  & \Gamma^1_{13} = \frac{1}{2}g^{k1}(\frac{\partial g_{k3}}{\partial X^1} + \frac{\partial g_{1k}}{\partial X^3} - \frac{\partial g_{13}}{\partial X^k}) = 0 \\
  & \Gamma^1_{21} = \frac{1}{2}g^{k1}(\frac{\partial g_{k1}}{\partial X^2} + \frac{\partial g_{2k}}{\partial X^1} - \frac{\partial g_{21}}{\partial X^k}) = 0 \\
  & \Gamma^1_{22} = \frac{1}{2}g^{k1}(\frac{\partial g_{k2}}{\partial X^2} + \frac{\partial g_{2k}}{\partial X^2} - \frac{\partial g_{22}}{\partial X^k}) = -1.0\cdot (X^1) \\
  & \Gamma^1_{23} = \frac{1}{2}g^{k1}(\frac{\partial g_{k3}}{\partial X^2} + \frac{\partial g_{2k}}{\partial X^3} - \frac{\partial g_{23}}{\partial X^k}) = 0 \\
  & \Gamma^1_{31} = \frac{1}{2}g^{k1}(\frac{\partial g_{k1}}{\partial X^3} + \frac{\partial g_{3k}}{\partial X^1} - \frac{\partial g_{31}}{\partial X^k}) = 0 \\
  & \Gamma^1_{32} = \frac{1}{2}g^{k1}(\frac{\partial g_{k2}}{\partial X^3} + \frac{\partial g_{3k}}{\partial X^2} - \frac{\partial g_{32}}{\partial X^k}) = 0 \\
  & \Gamma^1_{33} = \frac{1}{2}g^{k1}(\frac{\partial g_{k3}}{\partial X^3} + \frac{\partial g_{3k}}{\partial X^3} - \frac{\partial g_{33}}{\partial X^k}) = 0 
\end{alignat*}\\
При m = 2:\\
\begin{alignat*}{1}
  & \Gamma^2_{11} = \frac{1}{2}g^{k2}(\frac{\partial g_{k1}}{\partial X^1} + \frac{\partial g_{1k}}{\partial X^1} - \frac{\partial g_{11}}{\partial X^k}) = 0 \\
  & \Gamma^2_{12} = \frac{1}{2}g^{k2}(\frac{\partial g_{k2}}{\partial X^1} + \frac{\partial g_{1k}}{\partial X^2} - \frac{\partial g_{12}}{\partial X^k}) = 1.0/(X^1) \\
  & \Gamma^2_{13} = \frac{1}{2}g^{k2}(\frac{\partial g_{k3}}{\partial X^1} + \frac{\partial g_{1k}}{\partial X^3} - \frac{\partial g_{13}}{\partial X^k}) = 0 \\
  & \Gamma^2_{21} = \frac{1}{2}g^{k2}(\frac{\partial g_{k1}}{\partial X^2} + \frac{\partial g_{2k}}{\partial X^1} - \frac{\partial g_{21}}{\partial X^k}) = 1.0/(X^1) \\
  & \Gamma^2_{22} = \frac{1}{2}g^{k2}(\frac{\partial g_{k2}}{\partial X^2} + \frac{\partial g_{2k}}{\partial X^2} - \frac{\partial g_{22}}{\partial X^k}) = 0 \\
  & \Gamma^2_{23} = \frac{1}{2}g^{k2}(\frac{\partial g_{k3}}{\partial X^2} + \frac{\partial g_{2k}}{\partial X^3} - \frac{\partial g_{23}}{\partial X^k}) = 0 \\
  & \Gamma^2_{31} = \frac{1}{2}g^{k2}(\frac{\partial g_{k1}}{\partial X^3} + \frac{\partial g_{3k}}{\partial X^1} - \frac{\partial g_{31}}{\partial X^k}) = 0 \\
  & \Gamma^2_{32} = \frac{1}{2}g^{k2}(\frac{\partial g_{k2}}{\partial X^3} + \frac{\partial g_{3k}}{\partial X^2} - \frac{\partial g_{32}}{\partial X^k}) = 0 \\
  & \Gamma^2_{33} = \frac{1}{2}g^{k2}(\frac{\partial g_{k3}}{\partial X^3} + \frac{\partial g_{3k}}{\partial X^3} - \frac{\partial g_{33}}{\partial X^k}) = 0 
\end{alignat*}\\
При m = 3:\\
\begin{alignat*}{1}
  & \Gamma^3_{11} = \frac{1}{2}g^{k3}(\frac{\partial g_{k1}}{\partial X^1} + \frac{\partial g_{1k}}{\partial X^1} - \frac{\partial g_{11}}{\partial X^k}) = 0 \\
  & \Gamma^3_{12} = \frac{1}{2}g^{k3}(\frac{\partial g_{k2}}{\partial X^1} + \frac{\partial g_{1k}}{\partial X^2} - \frac{\partial g_{12}}{\partial X^k}) = 0 \\
  & \Gamma^3_{13} = \frac{1}{2}g^{k3}(\frac{\partial g_{k3}}{\partial X^1} + \frac{\partial g_{1k}}{\partial X^3} - \frac{\partial g_{13}}{\partial X^k}) = 0 \\
  & \Gamma^3_{21} = \frac{1}{2}g^{k3}(\frac{\partial g_{k1}}{\partial X^2} + \frac{\partial g_{2k}}{\partial X^1} - \frac{\partial g_{21}}{\partial X^k}) = 0 \\
  & \Gamma^3_{22} = \frac{1}{2}g^{k3}(\frac{\partial g_{k2}}{\partial X^2} + \frac{\partial g_{2k}}{\partial X^2} - \frac{\partial g_{22}}{\partial X^k}) = 0 \\
  & \Gamma^3_{23} = \frac{1}{2}g^{k3}(\frac{\partial g_{k3}}{\partial X^2} + \frac{\partial g_{2k}}{\partial X^3} - \frac{\partial g_{23}}{\partial X^k}) = 0 \\
  & \Gamma^3_{31} = \frac{1}{2}g^{k3}(\frac{\partial g_{k1}}{\partial X^3} + \frac{\partial g_{3k}}{\partial X^1} - \frac{\partial g_{31}}{\partial X^k}) = 0 \\
  & \Gamma^3_{32} = \frac{1}{2}g^{k3}(\frac{\partial g_{k2}}{\partial X^3} + \frac{\partial g_{3k}}{\partial X^2} - \frac{\partial g_{32}}{\partial X^k}) = 0 \\
  & \Gamma^3_{33} = \frac{1}{2}g^{k3}(\frac{\partial g_{k3}}{\partial X^3} + \frac{\partial g_{3k}}{\partial X^3} - \frac{\partial g_{33}}{\partial X^k}) = 0 
\end{alignat*}\\
Запишем результат:\\
\[
\Gamma^1_{ij} = \begin{pmatrix}
	0 & 0 & 0\\
	0 & -1.0\cdot (X^1) & 0\\
	0 & 0 & 0
\end{pmatrix}
\]\\
\[
\Gamma^2_{ij} = \begin{pmatrix}
	0 & 1.0/(X^1) & 0\\
	1.0/(X^1) & 0 & 0\\
	0 & 0 & 0
\end{pmatrix}
\]\\
\[
\Gamma^3_{ij} = \begin{pmatrix}
	0 & 0 & 0\\
	0 & 0 & 0\\
	0 & 0 & 0
\end{pmatrix}
\]\\
$\mathrm{2.1) }$Вычислим ковариантную производную контравариантных компонент поля по формуле:\\
\[
\nabla_kT^{ij} = \frac{\partial T^{ij}}{\partial X^k} + T^{mj}\Gamma^i_{mk} + T^{im}\Gamma^j_{mk};
\]\\
При k = 1:\\
\begin{alignat*}{1}
  & \nabla_1T^{11} = \frac{\partial T^{11}}{\partial X^1} + T^{m1}\Gamma^1_{m1} + T^{1m}\Gamma^1_{m1} = 0 \\
  & \nabla_1T^{12} = \frac{\partial T^{12}}{\partial X^1} + T^{m2}\Gamma^1_{m1} + T^{1m}\Gamma^2_{m1} = -1.0\cdot (X^2)/(X^1) + 2.0 \\
  & \nabla_1T^{13} = \frac{\partial T^{13}}{\partial X^1} + T^{m3}\Gamma^1_{m1} + T^{1m}\Gamma^3_{m1} = 0 \\
  & \nabla_1T^{21} = \frac{\partial T^{21}}{\partial X^1} + T^{m1}\Gamma^2_{m1} + T^{2m}\Gamma^1_{m1} = 0 \\
  & \nabla_1T^{22} = \frac{\partial T^{22}}{\partial X^1} + T^{m2}\Gamma^2_{m1} + T^{2m}\Gamma^2_{m1} = 0 \\
  & \nabla_1T^{23} = \frac{\partial T^{23}}{\partial X^1} + T^{m3}\Gamma^2_{m1} + T^{2m}\Gamma^3_{m1} = 0 \\
  & \nabla_1T^{31} = \frac{\partial T^{31}}{\partial X^1} + T^{m1}\Gamma^3_{m1} + T^{3m}\Gamma^1_{m1} = 0 \\
  & \nabla_1T^{32} = \frac{\partial T^{32}}{\partial X^1} + T^{m2}\Gamma^3_{m1} + T^{3m}\Gamma^2_{m1} = 0 \\
  & \nabla_1T^{33} = \frac{\partial T^{33}}{\partial X^1} + T^{m3}\Gamma^3_{m1} + T^{3m}\Gamma^3_{m1} = 0 
\end{alignat*}\\
При k = 2:\\
\begin{alignat*}{1}
  & \nabla_2T^{11} = \frac{\partial T^{11}}{\partial X^2} + T^{m1}\Gamma^1_{m2} + T^{1m}\Gamma^1_{m2} = 1.0\cdot (X^1)\cdot ((X^2) - (X^1)) \\
  & \nabla_2T^{12} = \frac{\partial T^{12}}{\partial X^2} + T^{m2}\Gamma^1_{m2} + T^{1m}\Gamma^2_{m2} = -1 \\
  & \nabla_2T^{13} = \frac{\partial T^{13}}{\partial X^2} + T^{m3}\Gamma^1_{m2} + T^{1m}\Gamma^3_{m2} = 0 \\
  & \nabla_2T^{21} = \frac{\partial T^{21}}{\partial X^2} + T^{m1}\Gamma^2_{m2} + T^{2m}\Gamma^1_{m2} = 0 \\
  & \nabla_2T^{22} = \frac{\partial T^{22}}{\partial X^2} + T^{m2}\Gamma^2_{m2} + T^{2m}\Gamma^2_{m2} = -1.0\cdot (X^2)/(X^1) + 1.0 \\
  & \nabla_2T^{23} = \frac{\partial T^{23}}{\partial X^2} + T^{m3}\Gamma^2_{m2} + T^{2m}\Gamma^3_{m2} = 0 \\
  & \nabla_2T^{31} = \frac{\partial T^{31}}{\partial X^2} + T^{m1}\Gamma^3_{m2} + T^{3m}\Gamma^1_{m2} = 0 \\
  & \nabla_2T^{32} = \frac{\partial T^{32}}{\partial X^2} + T^{m2}\Gamma^3_{m2} + T^{3m}\Gamma^2_{m2} = 2.0\cdot (X^3)/(X^1) \\
  & \nabla_2T^{33} = \frac{\partial T^{33}}{\partial X^2} + T^{m3}\Gamma^3_{m2} + T^{3m}\Gamma^3_{m2} = 0 
\end{alignat*}\\
При k = 3:\\
\begin{alignat*}{1}
  & \nabla_3T^{11} = \frac{\partial T^{11}}{\partial X^3} + T^{m1}\Gamma^1_{m3} + T^{1m}\Gamma^1_{m3} = 0 \\
  & \nabla_3T^{12} = \frac{\partial T^{12}}{\partial X^3} + T^{m2}\Gamma^1_{m3} + T^{1m}\Gamma^2_{m3} = 0 \\
  & \nabla_3T^{13} = \frac{\partial T^{13}}{\partial X^3} + T^{m3}\Gamma^1_{m3} + T^{1m}\Gamma^3_{m3} = 0 \\
  & \nabla_3T^{21} = \frac{\partial T^{21}}{\partial X^3} + T^{m1}\Gamma^2_{m3} + T^{2m}\Gamma^1_{m3} = 0 \\
  & \nabla_3T^{22} = \frac{\partial T^{22}}{\partial X^3} + T^{m2}\Gamma^2_{m3} + T^{2m}\Gamma^2_{m3} = 0 \\
  & \nabla_3T^{23} = \frac{\partial T^{23}}{\partial X^3} + T^{m3}\Gamma^2_{m3} + T^{2m}\Gamma^3_{m3} = 0 \\
  & \nabla_3T^{31} = \frac{\partial T^{31}}{\partial X^3} + T^{m1}\Gamma^3_{m3} + T^{3m}\Gamma^1_{m3} = 2 \\
  & \nabla_3T^{32} = \frac{\partial T^{32}}{\partial X^3} + T^{m2}\Gamma^3_{m3} + T^{3m}\Gamma^2_{m3} = 0 \\
  & \nabla_3T^{33} = \frac{\partial T^{33}}{\partial X^3} + T^{m3}\Gamma^3_{m3} + T^{3m}\Gamma^3_{m3} = 0 
\end{alignat*}\\
Запишем результат:\\
\[
\nabla_1T^{ij} = \begin{pmatrix}
	0 & -1.0\cdot (X^2)/(X^1) + 2.0 & 0\\
	0 & 0 & 0\\
	0 & 0 & 0
\end{pmatrix}
\]\\
\[
\nabla_2T^{ij} = \begin{pmatrix}
	1.0\cdot (X^1)\cdot ((X^2) - (X^1)) & -1 & 0\\
	0 & -1.0\cdot (X^2)/(X^1) + 1.0 & 0\\
	0 & 2.0\cdot (X^3)/(X^1) & 0
\end{pmatrix}
\]\\
\[
\nabla_3T^{ij} = \begin{pmatrix}
	0 & 0 & 0\\
	0 & 0 & 0\\
	2 & 0 & 0
\end{pmatrix}
\]\\
$\mathrm{2.2) }$Вычислим ковариантную производную ковариантных компонент поля по формуле:\\
\[
\nabla_kT_{ij} = \frac{\partial T_{ij}}{\partial X^k} - T_{mj}\Gamma^m_{ik} - T_{im}\Gamma^m_{jk};
\]\\
При k = 1:\\
\begin{alignat*}{1}
  & \nabla_1T_{11} = \frac{\partial T_{11}}{\partial X^1} - T_{m1}\Gamma^m_{11} - T_{1m}\Gamma^m_{11} = 0 \\
  & \nabla_1T_{12} = \frac{\partial T_{12}}{\partial X^1} - T_{m2}\Gamma^m_{11} - T_{1m}\Gamma^m_{21} = (X^1)\cdot (-1.0\cdot (X^2) + 2.0\cdot (X^1)) \\
  & \nabla_1T_{13} = \frac{\partial T_{13}}{\partial X^1} - T_{m3}\Gamma^m_{11} - T_{1m}\Gamma^m_{31} = 0 \\
  & \nabla_1T_{21} = \frac{\partial T_{21}}{\partial X^1} - T_{m1}\Gamma^m_{21} - T_{2m}\Gamma^m_{11} = 0 \\
  & \nabla_1T_{22} = \frac{\partial T_{22}}{\partial X^1} - T_{m2}\Gamma^m_{21} - T_{2m}\Gamma^m_{21} = 0 \\
  & \nabla_1T_{23} = \frac{\partial T_{23}}{\partial X^1} - T_{m3}\Gamma^m_{21} - T_{2m}\Gamma^m_{31} = 0 \\
  & \nabla_1T_{31} = \frac{\partial T_{31}}{\partial X^1} - T_{m1}\Gamma^m_{31} - T_{3m}\Gamma^m_{11} = 0 \\
  & \nabla_1T_{32} = \frac{\partial T_{32}}{\partial X^1} - T_{m2}\Gamma^m_{31} - T_{3m}\Gamma^m_{21} = 0 \\
  & \nabla_1T_{33} = \frac{\partial T_{33}}{\partial X^1} - T_{m3}\Gamma^m_{31} - T_{3m}\Gamma^m_{31} = 0 
\end{alignat*}\\
При k = 2:\\
\begin{alignat*}{1}
  & \nabla_2T_{11} = \frac{\partial T_{11}}{\partial X^2} - T_{m1}\Gamma^m_{12} - T_{1m}\Gamma^m_{12} = 1.0\cdot (X^1)\cdot ((X^2) - (X^1)) \\
  & \nabla_2T_{12} = \frac{\partial T_{12}}{\partial X^2} - T_{m2}\Gamma^m_{12} - T_{1m}\Gamma^m_{22} = -(X^1)^2 \\
  & \nabla_2T_{13} = \frac{\partial T_{13}}{\partial X^2} - T_{m3}\Gamma^m_{12} - T_{1m}\Gamma^m_{32} = 0 \\
  & \nabla_2T_{21} = \frac{\partial T_{21}}{\partial X^2} - T_{m1}\Gamma^m_{22} - T_{2m}\Gamma^m_{12} = 0 \\
  & \nabla_2T_{22} = \frac{\partial T_{22}}{\partial X^2} - T_{m2}\Gamma^m_{22} - T_{2m}\Gamma^m_{22} = 1.0\cdot (X^1)^3\cdot (-(X^2) + (X^1)) \\
  & \nabla_2T_{23} = \frac{\partial T_{23}}{\partial X^2} - T_{m3}\Gamma^m_{22} - T_{2m}\Gamma^m_{32} = 0 \\
  & \nabla_2T_{31} = \frac{\partial T_{31}}{\partial X^2} - T_{m1}\Gamma^m_{32} - T_{3m}\Gamma^m_{12} = 0 \\
  & \nabla_2T_{32} = \frac{\partial T_{32}}{\partial X^2} - T_{m2}\Gamma^m_{32} - T_{3m}\Gamma^m_{22} = 2.0\cdot (X^3)\cdot (X^1) \\
  & \nabla_2T_{33} = \frac{\partial T_{33}}{\partial X^2} - T_{m3}\Gamma^m_{32} - T_{3m}\Gamma^m_{32} = 0 
\end{alignat*}\\
При k = 3:\\
\begin{alignat*}{1}
  & \nabla_3T_{11} = \frac{\partial T_{11}}{\partial X^3} - T_{m1}\Gamma^m_{13} - T_{1m}\Gamma^m_{13} = 0 \\
  & \nabla_3T_{12} = \frac{\partial T_{12}}{\partial X^3} - T_{m2}\Gamma^m_{13} - T_{1m}\Gamma^m_{23} = 0 \\
  & \nabla_3T_{13} = \frac{\partial T_{13}}{\partial X^3} - T_{m3}\Gamma^m_{13} - T_{1m}\Gamma^m_{33} = 0 \\
  & \nabla_3T_{21} = \frac{\partial T_{21}}{\partial X^3} - T_{m1}\Gamma^m_{23} - T_{2m}\Gamma^m_{13} = 0 \\
  & \nabla_3T_{22} = \frac{\partial T_{22}}{\partial X^3} - T_{m2}\Gamma^m_{23} - T_{2m}\Gamma^m_{23} = 0 \\
  & \nabla_3T_{23} = \frac{\partial T_{23}}{\partial X^3} - T_{m3}\Gamma^m_{23} - T_{2m}\Gamma^m_{33} = 0 \\
  & \nabla_3T_{31} = \frac{\partial T_{31}}{\partial X^3} - T_{m1}\Gamma^m_{33} - T_{3m}\Gamma^m_{13} = 2 \\
  & \nabla_3T_{32} = \frac{\partial T_{32}}{\partial X^3} - T_{m2}\Gamma^m_{33} - T_{3m}\Gamma^m_{23} = 0 \\
  & \nabla_3T_{33} = \frac{\partial T_{33}}{\partial X^3} - T_{m3}\Gamma^m_{33} - T_{3m}\Gamma^m_{33} = 0 
\end{alignat*}\\
Запишем результат:\\
\[
\nabla_1T_{ij} = \begin{pmatrix}
	0 & (X^1)\cdot (-1.0\cdot (X^2) + 2.0\cdot (X^1)) & 0\\
	0 & 0 & 0\\
	0 & 0 & 0
\end{pmatrix}
\]\\
\[
\nabla_2T_{ij} = \begin{pmatrix}
	1.0\cdot (X^1)\cdot ((X^2) - (X^1)) & -(X^1)^2 & 0\\
	0 & 1.0\cdot (X^1)^3\cdot (-(X^2) + (X^1)) & 0\\
	0 & 2.0\cdot (X^3)\cdot (X^1) & 0
\end{pmatrix}
\]\\
\[
\nabla_3T_{ij} = \begin{pmatrix}
	0 & 0 & 0\\
	0 & 0 & 0\\
	2 & 0 & 0
\end{pmatrix}
\]\\
$\mathrm{2.3) }$Вычислим ковариантную производную смешанных компонент поля по формуле:\\
\[
\nabla_kT^i_j = \frac{\partial T^i_j}{\partial X^k} + T^m_j\Gamma^i_{mk} - T^i_m\Gamma^m_{jk};
\]\\
При k = 1:\\
\begin{alignat*}{1}
  & \nabla_1T^1_1 = \frac{\partial T^1_1}{\partial X^1} + T^m_1\Gamma^1_{m1} - T^1_m\Gamma^m_{11} = 0 \\
  & \nabla_1T^1_2 = \frac{\partial T^1_2}{\partial X^1} + T^m_2\Gamma^1_{m1} - T^1_m\Gamma^m_{21} = (X^1)\cdot (-1.0\cdot (X^2) + 2.0\cdot (X^1)) \\
  & \nabla_1T^1_3 = \frac{\partial T^1_3}{\partial X^1} + T^m_3\Gamma^1_{m1} - T^1_m\Gamma^m_{31} = 0 \\
  & \nabla_1T^2_1 = \frac{\partial T^2_1}{\partial X^1} + T^m_1\Gamma^2_{m1} - T^2_m\Gamma^m_{11} = 0 \\
  & \nabla_1T^2_2 = \frac{\partial T^2_2}{\partial X^1} + T^m_2\Gamma^2_{m1} - T^2_m\Gamma^m_{21} = 0 \\
  & \nabla_1T^2_3 = \frac{\partial T^2_3}{\partial X^1} + T^m_3\Gamma^2_{m1} - T^2_m\Gamma^m_{31} = 0 \\
  & \nabla_1T^3_1 = \frac{\partial T^3_1}{\partial X^1} + T^m_1\Gamma^3_{m1} - T^3_m\Gamma^m_{11} = 0 \\
  & \nabla_1T^3_2 = \frac{\partial T^3_2}{\partial X^1} + T^m_2\Gamma^3_{m1} - T^3_m\Gamma^m_{21} = 0 \\
  & \nabla_1T^3_3 = \frac{\partial T^3_3}{\partial X^1} + T^m_3\Gamma^3_{m1} - T^3_m\Gamma^m_{31} = 0 
\end{alignat*}\\
При k = 2:\\
\begin{alignat*}{1}
  & \nabla_2T^1_1 = \frac{\partial T^1_1}{\partial X^2} + T^m_1\Gamma^1_{m2} - T^1_m\Gamma^m_{12} = 1.0\cdot (X^1)\cdot ((X^2) - (X^1)) \\
  & \nabla_2T^1_2 = \frac{\partial T^1_2}{\partial X^2} + T^m_2\Gamma^1_{m2} - T^1_m\Gamma^m_{22} = -(X^1)^2 \\
  & \nabla_2T^1_3 = \frac{\partial T^1_3}{\partial X^2} + T^m_3\Gamma^1_{m2} - T^1_m\Gamma^m_{32} = 0 \\
  & \nabla_2T^2_1 = \frac{\partial T^2_1}{\partial X^2} + T^m_1\Gamma^2_{m2} - T^2_m\Gamma^m_{12} = 0 \\
  & \nabla_2T^2_2 = \frac{\partial T^2_2}{\partial X^2} + T^m_2\Gamma^2_{m2} - T^2_m\Gamma^m_{22} = 1.0\cdot (X^1)\cdot (-(X^2) + (X^1)) \\
  & \nabla_2T^2_3 = \frac{\partial T^2_3}{\partial X^2} + T^m_3\Gamma^2_{m2} - T^2_m\Gamma^m_{32} = 0 \\
  & \nabla_2T^3_1 = \frac{\partial T^3_1}{\partial X^2} + T^m_1\Gamma^3_{m2} - T^3_m\Gamma^m_{12} = 0 \\
  & \nabla_2T^3_2 = \frac{\partial T^3_2}{\partial X^2} + T^m_2\Gamma^3_{m2} - T^3_m\Gamma^m_{22} = 2.0\cdot (X^3)\cdot (X^1) \\
  & \nabla_2T^3_3 = \frac{\partial T^3_3}{\partial X^2} + T^m_3\Gamma^3_{m2} - T^3_m\Gamma^m_{32} = 0 
\end{alignat*}\\
При k = 3:\\
\begin{alignat*}{1}
  & \nabla_3T^1_1 = \frac{\partial T^1_1}{\partial X^3} + T^m_1\Gamma^1_{m3} - T^1_m\Gamma^m_{13} = 0 \\
  & \nabla_3T^1_2 = \frac{\partial T^1_2}{\partial X^3} + T^m_2\Gamma^1_{m3} - T^1_m\Gamma^m_{23} = 0 \\
  & \nabla_3T^1_3 = \frac{\partial T^1_3}{\partial X^3} + T^m_3\Gamma^1_{m3} - T^1_m\Gamma^m_{33} = 0 \\
  & \nabla_3T^2_1 = \frac{\partial T^2_1}{\partial X^3} + T^m_1\Gamma^2_{m3} - T^2_m\Gamma^m_{13} = 0 \\
  & \nabla_3T^2_2 = \frac{\partial T^2_2}{\partial X^3} + T^m_2\Gamma^2_{m3} - T^2_m\Gamma^m_{23} = 0 \\
  & \nabla_3T^2_3 = \frac{\partial T^2_3}{\partial X^3} + T^m_3\Gamma^2_{m3} - T^2_m\Gamma^m_{33} = 0 \\
  & \nabla_3T^3_1 = \frac{\partial T^3_1}{\partial X^3} + T^m_1\Gamma^3_{m3} - T^3_m\Gamma^m_{13} = 2 \\
  & \nabla_3T^3_2 = \frac{\partial T^3_2}{\partial X^3} + T^m_2\Gamma^3_{m3} - T^3_m\Gamma^m_{23} = 0 \\
  & \nabla_3T^3_3 = \frac{\partial T^3_3}{\partial X^3} + T^m_3\Gamma^3_{m3} - T^3_m\Gamma^m_{33} = 0 
\end{alignat*}\\
Запишем результат:\\
\[
\nabla_1T^i_j = \begin{pmatrix}
	0 & (X^1)\cdot (-1.0\cdot (X^2) + 2.0\cdot (X^1)) & 0\\
	0 & 0 & 0\\
	0 & 0 & 0
\end{pmatrix}
\]\\
\[
\nabla_2T^i_j = \begin{pmatrix}
	1.0\cdot (X^1)\cdot ((X^2) - (X^1)) & -(X^1)^2 & 0\\
	0 & 1.0\cdot (X^1)\cdot (-(X^2) + (X^1)) & 0\\
	0 & 2.0\cdot (X^3)\cdot (X^1) & 0
\end{pmatrix}
\]\\
\[
\nabla_3T^i_j = \begin{pmatrix}
	0 & 0 & 0\\
	0 & 0 & 0\\
	2 & 0 & 0
\end{pmatrix}
\]\\
$\mathrm{2.4) }$Вычислим ковариантную производную смешанных компонент поля по формуле:\\
\[
\nabla_kT^j_i = \frac{\partial T^j_i}{\partial X^k} - T^j_m\Gamma^m_{ik} + T^m_i\Gamma^j_{mk};
\]\\
При k = 1:\\
\begin{alignat*}{1}
  & \nabla_1T^1_1 = \frac{\partial T^1_1}{\partial X^1} - T^1_m\Gamma^m_{11} + T^m_1\Gamma^1_{m1} = 0 \\
  & \nabla_1T^2_1 = \frac{\partial T^2_1}{\partial X^1} - T^2_m\Gamma^m_{11} + T^m_1\Gamma^2_{m1} = -1.0\cdot (X^2)/(X^1) + 2.0 \\
  & \nabla_1T^3_1 = \frac{\partial T^3_1}{\partial X^1} - T^3_m\Gamma^m_{11} + T^m_1\Gamma^3_{m1} = 0 \\
  & \nabla_1T^1_2 = \frac{\partial T^1_2}{\partial X^1} - T^1_m\Gamma^m_{21} + T^m_2\Gamma^1_{m1} = 0 \\
  & \nabla_1T^2_2 = \frac{\partial T^2_2}{\partial X^1} - T^2_m\Gamma^m_{21} + T^m_2\Gamma^2_{m1} = 0 \\
  & \nabla_1T^3_2 = \frac{\partial T^3_2}{\partial X^1} - T^3_m\Gamma^m_{21} + T^m_2\Gamma^3_{m1} = 0 \\
  & \nabla_1T^1_3 = \frac{\partial T^1_3}{\partial X^1} - T^1_m\Gamma^m_{31} + T^m_3\Gamma^1_{m1} = 0 \\
  & \nabla_1T^2_3 = \frac{\partial T^2_3}{\partial X^1} - T^2_m\Gamma^m_{31} + T^m_3\Gamma^2_{m1} = 0 \\
  & \nabla_1T^3_3 = \frac{\partial T^3_3}{\partial X^1} - T^3_m\Gamma^m_{31} + T^m_3\Gamma^3_{m1} = 0 
\end{alignat*}\\
При k = 2:\\
\begin{alignat*}{1}
  & \nabla_2T^1_1 = \frac{\partial T^1_1}{\partial X^2} - T^1_m\Gamma^m_{12} + T^m_1\Gamma^1_{m2} = 1.0\cdot (X^1)\cdot ((X^2) - (X^1)) \\
  & \nabla_2T^2_1 = \frac{\partial T^2_1}{\partial X^2} - T^2_m\Gamma^m_{12} + T^m_1\Gamma^2_{m2} = -1 \\
  & \nabla_2T^3_1 = \frac{\partial T^3_1}{\partial X^2} - T^3_m\Gamma^m_{12} + T^m_1\Gamma^3_{m2} = 0 \\
  & \nabla_2T^1_2 = \frac{\partial T^1_2}{\partial X^2} - T^1_m\Gamma^m_{22} + T^m_2\Gamma^1_{m2} = 0 \\
  & \nabla_2T^2_2 = \frac{\partial T^2_2}{\partial X^2} - T^2_m\Gamma^m_{22} + T^m_2\Gamma^2_{m2} = 1.0\cdot (X^1)\cdot (-(X^2) + (X^1)) \\
  & \nabla_2T^3_2 = \frac{\partial T^3_2}{\partial X^2} - T^3_m\Gamma^m_{22} + T^m_2\Gamma^3_{m2} = 0 \\
  & \nabla_2T^1_3 = \frac{\partial T^1_3}{\partial X^2} - T^1_m\Gamma^m_{32} + T^m_3\Gamma^1_{m2} = 0 \\
  & \nabla_2T^2_3 = \frac{\partial T^2_3}{\partial X^2} - T^2_m\Gamma^m_{32} + T^m_3\Gamma^2_{m2} = 2.0\cdot (X^3)/(X^1) \\
  & \nabla_2T^3_3 = \frac{\partial T^3_3}{\partial X^2} - T^3_m\Gamma^m_{32} + T^m_3\Gamma^3_{m2} = 0 
\end{alignat*}\\
При k = 3:\\
\begin{alignat*}{1}
  & \nabla_3T^1_1 = \frac{\partial T^1_1}{\partial X^3} - T^1_m\Gamma^m_{13} + T^m_1\Gamma^1_{m3} = 0 \\
  & \nabla_3T^2_1 = \frac{\partial T^2_1}{\partial X^3} - T^2_m\Gamma^m_{13} + T^m_1\Gamma^2_{m3} = 0 \\
  & \nabla_3T^3_1 = \frac{\partial T^3_1}{\partial X^3} - T^3_m\Gamma^m_{13} + T^m_1\Gamma^3_{m3} = 0 \\
  & \nabla_3T^1_2 = \frac{\partial T^1_2}{\partial X^3} - T^1_m\Gamma^m_{23} + T^m_2\Gamma^1_{m3} = 0 \\
  & \nabla_3T^2_2 = \frac{\partial T^2_2}{\partial X^3} - T^2_m\Gamma^m_{23} + T^m_2\Gamma^2_{m3} = 0 \\
  & \nabla_3T^3_2 = \frac{\partial T^3_2}{\partial X^3} - T^3_m\Gamma^m_{23} + T^m_2\Gamma^3_{m3} = 0 \\
  & \nabla_3T^1_3 = \frac{\partial T^1_3}{\partial X^3} - T^1_m\Gamma^m_{33} + T^m_3\Gamma^1_{m3} = 2 \\
  & \nabla_3T^2_3 = \frac{\partial T^2_3}{\partial X^3} - T^2_m\Gamma^m_{33} + T^m_3\Gamma^2_{m3} = 0 \\
  & \nabla_3T^3_3 = \frac{\partial T^3_3}{\partial X^3} - T^3_m\Gamma^m_{33} + T^m_3\Gamma^3_{m3} = 0 
\end{alignat*}\\
Запишем результат:\\
\[
\nabla_1T^j_i = \begin{pmatrix}
	0 & -1.0\cdot (X^2)/(X^1) + 2.0 & 0\\
	0 & 0 & 0\\
	0 & 0 & 0
\end{pmatrix}
\]\\
\[
\nabla_2T^j_i = \begin{pmatrix}
	1.0\cdot (X^1)\cdot ((X^2) - (X^1)) & -1 & 0\\
	0 & 1.0\cdot (X^1)\cdot (-(X^2) + (X^1)) & 0\\
	0 & 2.0\cdot (X^3)/(X^1) & 0
\end{pmatrix}
\]\\
\[
\nabla_3T^j_i = \begin{pmatrix}
	0 & 0 & 0\\
	0 & 0 & 0\\
	2 & 0 & 0
\end{pmatrix}
\]\\
Вычислим контравариантные производные.\\
$\mathrm{2.5) }$Вычислим контравариантную производную от контравариантных компонент по формуле:\\
\[
\nabla^mT^{ij} = g^{mk}\nabla_kT^{ij};
\]\\
При k = 1:\\
\begin{alignat*}{1}
  & \nabla^1T^{11} = g^{1k}\nabla_kT^{11} = 0 \\
  & \nabla^1T^{12} = g^{1k}\nabla_kT^{12} = -1.0\cdot (X^2)/(X^1) + 2.0 \\
  & \nabla^1T^{13} = g^{1k}\nabla_kT^{13} = 0 \\
  & \nabla^1T^{21} = g^{1k}\nabla_kT^{21} = 0 \\
  & \nabla^1T^{22} = g^{1k}\nabla_kT^{22} = 0 \\
  & \nabla^1T^{23} = g^{1k}\nabla_kT^{23} = 0 \\
  & \nabla^1T^{31} = g^{1k}\nabla_kT^{31} = 0 \\
  & \nabla^1T^{32} = g^{1k}\nabla_kT^{32} = 0 \\
  & \nabla^1T^{33} = g^{1k}\nabla_kT^{33} = 0 
\end{alignat*}\\
При k = 2:\\
\begin{alignat*}{1}
  & \nabla^2T^{11} = g^{2k}\nabla_kT^{11} = 1.0\cdot (X^2)/(X^1) - 1.0 \\
  & \nabla^2T^{12} = g^{2k}\nabla_kT^{12} = -1/(X^1)^2 \\
  & \nabla^2T^{13} = g^{2k}\nabla_kT^{13} = 0 \\
  & \nabla^2T^{21} = g^{2k}\nabla_kT^{21} = 0 \\
  & \nabla^2T^{22} = g^{2k}\nabla_kT^{22} = 1.0\cdot (-(X^2) + (X^1))/(X^1)^3 \\
  & \nabla^2T^{23} = g^{2k}\nabla_kT^{23} = 0 \\
  & \nabla^2T^{31} = g^{2k}\nabla_kT^{31} = 0 \\
  & \nabla^2T^{32} = g^{2k}\nabla_kT^{32} = 2.0\cdot (X^3)/(X^1)^3 \\
  & \nabla^2T^{33} = g^{2k}\nabla_kT^{33} = 0 
\end{alignat*}\\
При k = 3:\\
\begin{alignat*}{1}
  & \nabla^3T^{11} = g^{3k}\nabla_kT^{11} = 0 \\
  & \nabla^3T^{12} = g^{3k}\nabla_kT^{12} = 0 \\
  & \nabla^3T^{13} = g^{3k}\nabla_kT^{13} = 0 \\
  & \nabla^3T^{21} = g^{3k}\nabla_kT^{21} = 0 \\
  & \nabla^3T^{22} = g^{3k}\nabla_kT^{22} = 0 \\
  & \nabla^3T^{23} = g^{3k}\nabla_kT^{23} = 0 \\
  & \nabla^3T^{31} = g^{3k}\nabla_kT^{31} = 2 \\
  & \nabla^3T^{32} = g^{3k}\nabla_kT^{32} = 0 \\
  & \nabla^3T^{33} = g^{3k}\nabla_kT^{33} = 0 
\end{alignat*}\\
Запишем результат:\\
\[
\nabla^1T^{ij} = \begin{pmatrix}
	0 & -1.0\cdot (X^2)/(X^1) + 2.0 & 0\\
	0 & 0 & 0\\
	0 & 0 & 0
\end{pmatrix}
\]\\
\[
\nabla^2T^{ij} = \begin{pmatrix}
	1.0\cdot (X^2)/(X^1) - 1.0 & -1/(X^1)^2 & 0\\
	0 & 1.0\cdot (-(X^2) + (X^1))/(X^1)^3 & 0\\
	0 & 2.0\cdot (X^3)/(X^1)^3 & 0
\end{pmatrix}
\]\\
\[
\nabla^3T^{ij} = \begin{pmatrix}
	0 & 0 & 0\\
	0 & 0 & 0\\
	2 & 0 & 0
\end{pmatrix}
\]\\
$\mathrm{2.6) }$Вычислим контравариантную производную от ковариантных компонент по формуле:\\
\[
\nabla^mT_{ij} = g^{mk}\nabla_kT_{ij};
\]\\
При k = 1:\\
\begin{alignat*}{1}
  & \nabla^1T_{11} = g^{1k}\nabla_kT_{11} = 0 \\
  & \nabla^1T_{12} = g^{1k}\nabla_kT_{12} = (X^1)\cdot (-1.0\cdot (X^2) + 2.0\cdot (X^1)) \\
  & \nabla^1T_{13} = g^{1k}\nabla_kT_{13} = 0 \\
  & \nabla^1T_{21} = g^{1k}\nabla_kT_{21} = 0 \\
  & \nabla^1T_{22} = g^{1k}\nabla_kT_{22} = 0 \\
  & \nabla^1T_{23} = g^{1k}\nabla_kT_{23} = 0 \\
  & \nabla^1T_{31} = g^{1k}\nabla_kT_{31} = 0 \\
  & \nabla^1T_{32} = g^{1k}\nabla_kT_{32} = 0 \\
  & \nabla^1T_{33} = g^{1k}\nabla_kT_{33} = 0 
\end{alignat*}\\
При k = 2:\\
\begin{alignat*}{1}
  & \nabla^2T_{11} = g^{2k}\nabla_kT_{11} = 1.0\cdot (X^2)/(X^1) - 1.0 \\
  & \nabla^2T_{12} = g^{2k}\nabla_kT_{12} = -1 \\
  & \nabla^2T_{13} = g^{2k}\nabla_kT_{13} = 0 \\
  & \nabla^2T_{21} = g^{2k}\nabla_kT_{21} = 0 \\
  & \nabla^2T_{22} = g^{2k}\nabla_kT_{22} = 1.0\cdot (X^1)\cdot (-(X^2) + (X^1)) \\
  & \nabla^2T_{23} = g^{2k}\nabla_kT_{23} = 0 \\
  & \nabla^2T_{31} = g^{2k}\nabla_kT_{31} = 0 \\
  & \nabla^2T_{32} = g^{2k}\nabla_kT_{32} = 2.0\cdot (X^3)/(X^1) \\
  & \nabla^2T_{33} = g^{2k}\nabla_kT_{33} = 0 
\end{alignat*}\\
При k = 3:\\
\begin{alignat*}{1}
  & \nabla^3T_{11} = g^{3k}\nabla_kT_{11} = 0 \\
  & \nabla^3T_{12} = g^{3k}\nabla_kT_{12} = 0 \\
  & \nabla^3T_{13} = g^{3k}\nabla_kT_{13} = 0 \\
  & \nabla^3T_{21} = g^{3k}\nabla_kT_{21} = 0 \\
  & \nabla^3T_{22} = g^{3k}\nabla_kT_{22} = 0 \\
  & \nabla^3T_{23} = g^{3k}\nabla_kT_{23} = 0 \\
  & \nabla^3T_{31} = g^{3k}\nabla_kT_{31} = 2 \\
  & \nabla^3T_{32} = g^{3k}\nabla_kT_{32} = 0 \\
  & \nabla^3T_{33} = g^{3k}\nabla_kT_{33} = 0 
\end{alignat*}\\
Запишем результат:\\
\[
\nabla^1T_{ij} = \begin{pmatrix}
	0 & (X^1)\cdot (-1.0\cdot (X^2) + 2.0\cdot (X^1)) & 0\\
	0 & 0 & 0\\
	0 & 0 & 0
\end{pmatrix}
\]\\
\[
\nabla^2T_{ij} = \begin{pmatrix}
	1.0\cdot (X^2)/(X^1) - 1.0 & -1 & 0\\
	0 & 1.0\cdot (X^1)\cdot (-(X^2) + (X^1)) & 0\\
	0 & 2.0\cdot (X^3)/(X^1) & 0
\end{pmatrix}
\]\\
\[
\nabla^3T_{ij} = \begin{pmatrix}
	0 & 0 & 0\\
	0 & 0 & 0\\
	2 & 0 & 0
\end{pmatrix}
\]\\
$\mathrm{2.7) }$Аналогично вычислим контравариантную производную от смешанных компонент тензора по формуле:\\
\[
\nabla^mT^i_j = g^{mk}\nabla_kT^i_j;
\]\\
При k = 1:\\
\begin{alignat*}{1}
  & \nabla^1T^1_1 = g^{1k}\nabla_kT^1_1 = 0 \\
  & \nabla^1T^1_2 = g^{1k}\nabla_kT^1_2 = (X^1)\cdot (-1.0\cdot (X^2) + 2.0\cdot (X^1)) \\
  & \nabla^1T^1_3 = g^{1k}\nabla_kT^1_3 = 0 \\
  & \nabla^1T^2_1 = g^{1k}\nabla_kT^2_1 = 0 \\
  & \nabla^1T^2_2 = g^{1k}\nabla_kT^2_2 = 0 \\
  & \nabla^1T^2_3 = g^{1k}\nabla_kT^2_3 = 0 \\
  & \nabla^1T^3_1 = g^{1k}\nabla_kT^3_1 = 0 \\
  & \nabla^1T^3_2 = g^{1k}\nabla_kT^3_2 = 0 \\
  & \nabla^1T^3_3 = g^{1k}\nabla_kT^3_3 = 0 
\end{alignat*}\\
При k = 2:\\
\begin{alignat*}{1}
  & \nabla^2T^1_1 = g^{2k}\nabla_kT^1_1 = 1.0\cdot (X^2)/(X^1) - 1.0 \\
  & \nabla^2T^1_2 = g^{2k}\nabla_kT^1_2 = -1 \\
  & \nabla^2T^1_3 = g^{2k}\nabla_kT^1_3 = 0 \\
  & \nabla^2T^2_1 = g^{2k}\nabla_kT^2_1 = 0 \\
  & \nabla^2T^2_2 = g^{2k}\nabla_kT^2_2 = -1.0\cdot (X^2)/(X^1) + 1.0 \\
  & \nabla^2T^2_3 = g^{2k}\nabla_kT^2_3 = 0 \\
  & \nabla^2T^3_1 = g^{2k}\nabla_kT^3_1 = 0 \\
  & \nabla^2T^3_2 = g^{2k}\nabla_kT^3_2 = 2.0\cdot (X^3)/(X^1) \\
  & \nabla^2T^3_3 = g^{2k}\nabla_kT^3_3 = 0 
\end{alignat*}\\
При k = 3:\\
\begin{alignat*}{1}
  & \nabla^3T^1_1 = g^{3k}\nabla_kT^1_1 = 0 \\
  & \nabla^3T^1_2 = g^{3k}\nabla_kT^1_2 = 0 \\
  & \nabla^3T^1_3 = g^{3k}\nabla_kT^1_3 = 0 \\
  & \nabla^3T^2_1 = g^{3k}\nabla_kT^2_1 = 0 \\
  & \nabla^3T^2_2 = g^{3k}\nabla_kT^2_2 = 0 \\
  & \nabla^3T^2_3 = g^{3k}\nabla_kT^2_3 = 0 \\
  & \nabla^3T^3_1 = g^{3k}\nabla_kT^3_1 = 2 \\
  & \nabla^3T^3_2 = g^{3k}\nabla_kT^3_2 = 0 \\
  & \nabla^3T^3_3 = g^{3k}\nabla_kT^3_3 = 0 
\end{alignat*}\\
Запишем результат:\\
\[
\nabla^1T^i_j = \begin{pmatrix}
	0 & (X^1)\cdot (-1.0\cdot (X^2) + 2.0\cdot (X^1)) & 0\\
	0 & 0 & 0\\
	0 & 0 & 0
\end{pmatrix}
\]\\
\[
\nabla^2T^i_j = \begin{pmatrix}
	1.0\cdot (X^2)/(X^1) - 1.0 & -1 & 0\\
	0 & -1.0\cdot (X^2)/(X^1) + 1.0 & 0\\
	0 & 2.0\cdot (X^3)/(X^1) & 0
\end{pmatrix}
\]\\
\[
\nabla^3T^i_j = \begin{pmatrix}
	0 & 0 & 0\\
	0 & 0 & 0\\
	2 & 0 & 0
\end{pmatrix}
\]\\
$\mathrm{2.8) }$Аналогично вычислим контравариантную производную от смешанных компонент тензора по формуле:\\
\[
\nabla^mT^j_i = g^{mk}\nabla_kT^j_i;
\]\\
При k = 1:\\
\begin{alignat*}{1}
  & \nabla^1T^1_1 = g^{1k}\nabla_kT^1_1 = 0 \\
  & \nabla^1T^2_1 = g^{1k}\nabla_kT^2_1 = -1.0\cdot (X^2)/(X^1) + 2.0 \\
  & \nabla^1T^3_1 = g^{1k}\nabla_kT^3_1 = 0 \\
  & \nabla^1T^1_2 = g^{1k}\nabla_kT^1_2 = 0 \\
  & \nabla^1T^2_2 = g^{1k}\nabla_kT^2_2 = 0 \\
  & \nabla^1T^3_2 = g^{1k}\nabla_kT^3_2 = 0 \\
  & \nabla^1T^1_3 = g^{1k}\nabla_kT^1_3 = 0 \\
  & \nabla^1T^2_3 = g^{1k}\nabla_kT^2_3 = 0 \\
  & \nabla^1T^3_3 = g^{1k}\nabla_kT^3_3 = 0 
\end{alignat*}\\
При k = 2:\\
\begin{alignat*}{1}
  & \nabla^2T^1_1 = g^{2k}\nabla_kT^1_1 = 1.0\cdot (X^2)/(X^1) - 1.0 \\
  & \nabla^2T^2_1 = g^{2k}\nabla_kT^2_1 = -1/(X^1)^2 \\
  & \nabla^2T^3_1 = g^{2k}\nabla_kT^3_1 = 0 \\
  & \nabla^2T^1_2 = g^{2k}\nabla_kT^1_2 = 0 \\
  & \nabla^2T^2_2 = g^{2k}\nabla_kT^2_2 = -1.0\cdot (X^2)/(X^1) + 1.0 \\
  & \nabla^2T^3_2 = g^{2k}\nabla_kT^3_2 = 0 \\
  & \nabla^2T^1_3 = g^{2k}\nabla_kT^1_3 = 0 \\
  & \nabla^2T^2_3 = g^{2k}\nabla_kT^2_3 = 2.0\cdot (X^3)/(X^1)^3 \\
  & \nabla^2T^3_3 = g^{2k}\nabla_kT^3_3 = 0 
\end{alignat*}\\
При k = 3:\\
\begin{alignat*}{1}
  & \nabla^3T^1_1 = g^{3k}\nabla_kT^1_1 = 0 \\
  & \nabla^3T^2_1 = g^{3k}\nabla_kT^2_1 = 0 \\
  & \nabla^3T^3_1 = g^{3k}\nabla_kT^3_1 = 0 \\
  & \nabla^3T^1_2 = g^{3k}\nabla_kT^1_2 = 0 \\
  & \nabla^3T^2_2 = g^{3k}\nabla_kT^2_2 = 0 \\
  & \nabla^3T^3_2 = g^{3k}\nabla_kT^3_2 = 0 \\
  & \nabla^3T^1_3 = g^{3k}\nabla_kT^1_3 = 2 \\
  & \nabla^3T^2_3 = g^{3k}\nabla_kT^2_3 = 0 \\
  & \nabla^3T^3_3 = g^{3k}\nabla_kT^3_3 = 0 
\end{alignat*}\\
Запишем результат:\\
\[
\nabla^1T^j_i = \begin{pmatrix}
	0 & -1.0\cdot (X^2)/(X^1) + 2.0 & 0\\
	0 & 0 & 0\\
	0 & 0 & 0
\end{pmatrix}
\]\\
\[
\nabla^2T^j_i = \begin{pmatrix}
	1.0\cdot (X^2)/(X^1) - 1.0 & -1/(X^1)^2 & 0\\
	0 & -1.0\cdot (X^2)/(X^1) + 1.0 & 0\\
	0 & 2.0\cdot (X^3)/(X^1)^3 & 0
\end{pmatrix}
\]\\
\[
\nabla^3T^j_i = \begin{pmatrix}
	0 & 0 & 0\\
	0 & 0 & 0\\
	2 & 0 & 0
\end{pmatrix}
\]\\
\end{document}
